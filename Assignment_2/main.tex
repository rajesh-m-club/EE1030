\let\negmedspace\undefined
\let\negthickspace\undefined
\documentclass[journal,12pt,onecolumn,article]{IEEEtran}
\usepackage{cite}
\usepackage{color,soul}
\usepackage{amsmath,amssymb,amsfonts,amsthm}
\usepackage{algorithmic}
\usepackage{graphicx}
\usepackage{textcomp}
\usepackage{xcolor}
\usepackage{txfonts}
\usepackage{listings}
\usepackage{enumitem}
\usepackage{mathtools}
\usepackage{gensymb}
\usepackage{comment}
\usepackage[breaklinks=true]{hyperref}
\usepackage{tkz-euclide} 
\usepackage{listings}
\usepackage{multicol}
\usepackage{gvv}       
\usepackage[dvipsnames]{xcolor}
\def\inputGnumericTable{}                                
\usepackage[latin1]{inputenc}                            
\usepackage{color}                                       
\usepackage{array}                                       
\usepackage{longtable}                                   
\usepackage{calc}                                        
\usepackage{multirow}                                    
\usepackage{hhline}                                      
\usepackage{ifthen}                                      
\usepackage{lscape}
\newtheorem{theorem}{Theorem}[section]
\newtheorem{problem}{Problem}
\newtheorem{proposition}{Proposition}[section]
\newtheorem{lemma}{Lemma}[section]
\newtheorem{corollary}[theorem]{Corollary}
\newtheorem{example}{Example}[section]
\newtheorem{definition}[problem]{Definition}
\newcommand{\BEQA}{\begin{eqnarray}}
\newcommand{\EEQA}{\end{eqnarray}}
\newcommand{\define}{\stackrel{\triangle}{=}}
\theoremstyle{remark}
\newtheorem{rem}{Remark}
\begin{document}
\bibliographystyle{IEEEtran}
\vspace{3cm}
\title{ASSIGNMENT-2}
\author{EE24BTECH11043 - Murra Rajesh Kumar Reddy}
\maketitle
\bigskip
\begin{enumerate}
\item $A$ is targeting $B$,$B$ and $C$ are targeting to $A$.Probability of hitting the target by $A,B$ and $C$ are $\frac{2}{3},\frac{1}{2}$ and $\frac{1}{3}$ respectively. If $A$ is hit then find the probability that $B$ hits the target and $C$ does not. \hfill{$\brak{2003-2 Marks}$}
\item $A$ and $B$ are two independent events. $C$ is event in which exactly one of $A$ or $B$ occurs. Prove that $P\brak{C}\ge P\brak{A\cup B}P\brak{\overline{A}\cap\overline{B}}$ \hfill{$\brak{2004-2 Marks}$}
\item A box contains $12$ red and $6$ white balls.Balls are drawn from the box one at a time without replacement. If in 6 draws there are at least $4$ white balls, find the probability that exactly one white drawn in the next two draws. $($
binomial coefficients can be left as such$)$ \hfill{$\brak{2004-4 Marks}$}
\item A person goes to office either by car, scooter,bus or train the probability of which being $\frac{1}{7},\frac{3}{7},\frac{3}{7},\frac{2}{7},$ and$\frac{1}{7}$ respectively. Probability that he reachs office late, ife takes car,scooter,bus or train is $\frac{2}{9},\frac{1}{9},\frac{4}{9}$ and $\frac{1}{9}$ respectively. Goven that he reached office in time, then what is the probability that he travelled by a car. \hfill{$\brak{2005-2 Marks}$}
\end{enumerate}
\section*{G Comprehension Based Questions}
\section*{PASSAGE-1}
There are n urns, each of these contain n+1 balls. The $i^{th}$ urn contains $i$white balls and $\brak{n+1-i}$ red balls. Let $u_1$ be the event of selecting $i^{th}$ urn, $i=1,2,3\dots,n$ and $w$ the event of getting a white ball.
\begin{enumerate}
	\item If $P\brak{u_i}\propto i,$ where $i=1,2,3,\dots,n,$ then $\lim_{n \to \infty} P\brak{w}=$ \hfill{$\brak{2006-5M,-2}$}
		\begin{enumerate}
				\begin{multicols}{4}
					\item $1$
						\columnbreak
					\item $\frac{2}{3}$
						\columnbreak
					\item $\frac{3}{4}$
						\columnbreak
					\item $\frac{1}{4}$
				\end{multicols}
		\end{enumerate}
	\item If $P\brak{u_i}=c$, $($a constant$)$ then $P\brak{\frac{u_n}{w}}=$\hfill{$\brak{2006-5M,-2}$}
		\begin{enumerate}
				\begin{multicols}{4}
				\item $\frac{2}{n+1}$ \columnbreak
				\item $\frac{1}{n+1}$ \columnbreak
				\item $\frac{n}{n+1}$ \columnbreak
				\item $\frac{1}{2}$
				\end{multicols}
		\end{enumerate}
	\item Let $P\brak{u_i}=\frak{1}{n}$, if $n$ is even and $E$ denotes the event of choosing even numbered urn, then the value of $P\brak{\frac{w}{E}}$ is \hfill{$\brak{2006-5M,-2}$}
		\begin{enumerate}
				\begin{multicols}{4}
				\item $\frac{n+2}{2n+1}$ \columnbreak
				\item $\frac{n+2}{2\brak{n+1}}$ \columnbreak
				\item $\frac{n}{n+1}$ \columnbreak 
				\item $\frac{1}{n+1}$
				\end{multicols}
		\end{enumerate}
		\section*{PASSAGE-2}
A fair die is tossed repeatedly until a six is obtained. Let $X$ denote the number of tosses required. \hfill{$\brak{2009}$}
         \item The probability that $X=3$ equals
		 \begin{enumerate}
				 \begin{multicols}{4}
				 \item $\frac{25}{216}$ \columnbreak
				 \item $\frac{25}{36}$ \columnbreak
				 \item $\frac{5}{36}$ \columnbreak
				 \item $\frac{125}{216}$
				 \end{multicols}
		 \end{enumerate}
	 \item The probability that $X\ge3$ equals 
		 \begin{enumerate}
				 \begin{multicols}{4}
				 \item $\frac{125}{216}$ \columnbreak
				 \item $\frac{25}{216}$ \columnbreak
				 \item $\frac{5}{36}$ \columnbreak
				 \item $\frac{25}{36}$
				 \end{multicols}
		 \end{enumerate}
	 \item The conditional probability that $X\ge6$ given $X>3$ equals
		 \begin{enumerate}
				 \begin{multicols}{4}
				 \item $\frac{125}{216}$ \columnbreak
				 \item $\frac{25}{216}$ \columnbreak
				 \item $\frac{5}{36}$ \columnbreak
				 \item $\frac{25}{36}$
				 \end{multicols}
		 \end{enumerate}
		 \section*{PASSAGE-3}
Let $U_1$ and $U_2$ be two urns such that $U_1$ contains $3$ white and $2$ red balls, and $U_2$ contains only $1$ white ball.A fair coin is tossed.If head appears then $1$ ball is drawn at random from $U_1$ and put into $U_2$.However, if tail appears then $2$ balls are drawn at random from $U_1$ and put into $U_2$. Now $1$ ball is drawn at random from $U_2$. \hfill{$\brak{2011}$}
         \item The probability of the drawn ball from $U_2$ being white is
		 \begin{enumerate}
				 \begin{multicols}{4}
				 \item $\frac{13}{30}$ \columnbreak
				 \item $\frac{23}{30}$ \columnbreak
				 \item $\frac{19}{30}$ \columnbreak
				 \item $\frac{11}{30}$
				 \end{multicols}
		 \end{enumerate}
	 \item Given that the drawn ball from $U_2$ is white, the probability that head appeared on the coin is 
		 \begin{enumerate}
				 \begin{multicols}{4}
				 \item $\frac{17}{23}$ \columnbreak
				 \item $\frac{11}{23}$ \columnbreak
				 \item $\frac{15}{23}$ \columnbreak
				 \item $\frac{12}{23}$
				 \end{multicols}
		 \end{enumerate}
\end{enumerate}
\end{document}

