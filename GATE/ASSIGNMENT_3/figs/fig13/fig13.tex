\documentclass{article}
\usepackage{circuitikz}
\usepackage{amsmath}

\begin{document}

\begin{figure}[!ht]
\centering
\resizebox{1\textwidth}{!}{%
\begin{circuitikz}
\tikzstyle{every node}=[font=\Large]

\draw [line width=1.5pt, short] (0,9) -- (21,9);
\draw [line width=1.4pt, dashed] (10.5,15.75) -- (10.75,1.5);
\draw [line width=1.2pt, short] (6.25,13.5) -- (10.5,9);
\draw [short] (10.5,9) -- (12.5,9);
\draw [line width=1.2pt, short] (10.5,9) -- (13.5,3);
\draw [line width=1.2pt, ->, >=Stealth] (5,14.75) -- (6.25,13.5);
\draw [line width=1.3pt, ->, >=Stealth] (13.5,3) -- (14.25,1.5);
\draw [line width=1pt, short] (9.75,9.75) .. controls (10,10.25) and (10.25,10.5) .. (10.5,10.25);
\draw [line width=1pt, short] (10.75,6.75) .. controls (11.25,6.75) and (11.25,6.75) .. (11.5,7);

\node [font=\Large] at (20.75,9.75) {$\epsilon_2$};
\node [font=\Large] at (20.5,8.5) {$\epsilon_2$};
\node [font=\Large] at (10,10.75) {$\theta_1$};
\node [font=\Large] at (11.25,6.25) {$\theta_2$};
\node [font=\Large] at (4.75,14) {$E_1$};
\node [font=\Large] at (14.75,2.5) {$E_2$};

\end{circuitikz}
}%

\end{figure}

\end{document}

