\let\negmedspace\undefined
\let\negthickspace\undefined
\documentclass[journal,12pt,onecolumn,article]{IEEEtran}
\usepackage{cite}
\usepackage{color,soul}
\usepackage{amsmath,amssymb,amsfonts,amsthm}
\usepackage{algorithmic}
\usepackage{graphicx}
\usepackage{textcomp}
\usepackage{xcolor}
\usepackage{txfonts}
\usepackage{listings}
\usepackage{enumitem}
\usepackage{mathtools}
\usepackage{gensymb}
\usepackage{comment}
\usepackage[breaklinks=true]{hyperref}
\usepackage{tkz-euclide} 
\usepackage{listings}
\usepackage{multicol}
\usepackage{gvv}       
\usepackage[dvipsnames]{xcolor}
\def\inputGnumericTable{}                                
\usepackage[latin1]{inputenc}                            
\usepackage{color}                                       
\usepackage{array}                                       
\usepackage{longtable}                                   
\usepackage{calc}                                        
\usepackage{multirow}                                    
\usepackage{hhline}                                      
\usepackage{ifthen}                                      
\usepackage{lscape}
\newtheorem{theorem}{Theorem}[section]
\newtheorem{problem}{Problem}
\newtheorem{proposition}{Proposition}[section]
\newtheorem{lemma}{Lemma}[section]
\newtheorem{corollary}[theorem]{Corollary}
\newtheorem{example}{Example}[section]
\newtheorem{definition}[problem]{Definition}
\newcommand{\BEQA}{\begin{eqnarray}}
\newcommand{\EEQA}{\end{eqnarray}}
\newcommand{\define}{\stackrel{\triangle}{=}}
\theoremstyle{remark}
\newtheorem{rem}{Remark}
\begin{document}
\bibliographystyle{IEEEtran}
\vspace{3cm}
\title{ASSIGNMENT-2}
\author{EE24BTECH11043 - Murra Rajesh Kumar Reddy}
\maketitle
\bigskip
\begin{enumerate}
\item The shortest distance between the lines $\frac{x+7}{-6} = \frac{y-6}{7} = z$ and $\frac{7-x}{2} = y-2 = z-6$ is 
\begin{enumerate}
\item $2\sqrt{29}$
\item $1$
\item $\sqrt{\frac{37}{29}}$
\item $\sqrt{\frac{29}{2}}$
\end{enumerate}
\item Let $\vec{a} = \hat{i}-\hat{j}+2\hat{k}$ and let $\vec{b}$ be a vector such that $\vec{a} \times \vec{b} = 2\hat{i}-\hat{k}$ and $\vec{a} \cdot \vec{b} = 3$ Then the projection of $\vec{b}$ on the vector $\vec{a}-\vec{b}$ is :
\begin{enumerate}
\item $\frac{2}{\sqrt{21}}$
\item $2\sqrt{\frac{3}{7}}$
\item $\frac{2}{3}\sqrt{\frac{7}{3}}$
\item $\frac{2}{3}$
\end{enumerate}
\item If the mean deviation about median for the number $3,5,7,2k,12,16,21,24$ arranged in the ascending order, is 6 then the median is
\begin{enumerate}
\item $11.5$
\item $10.5$
\item $12$
\item $11$
\end{enumerate}
\item $2\sin\brak{\frac{\pi}{22}}\sin\brak{\frac{3\pi}{22}}\sin\brak{\frac{5\pi}{22}}\sin\brak{\frac{7\pi}{22}}\sin\brak{\frac{9\pi}{22}}$ is eqaul to :
\begin{enumerate}
\item $\frac{3}{16}$
\item $\frac{1}{16}$
\item $\frac{1}{32}$
\item $\frac{9}{32}$
\end{enumerate}
\item Consider the following statements : \\
$P$ : Ramu is intelligent. \\
$Q$ : Ramu is rich. \\
$R$ : Ramu is not honest.\\
The negation of the statement "Ramu is intelligent and honest if and only if Ramu is not rich" can be expressed as : 
\begin{enumerate}
\item $\brak{\brak{P \cap \brak{\sim R}}\cap Q} \brak{\brak{\sim Q} \cap \brak{\brak{\sim P} \cup R}}$
\item $\brak{\brak{P \cap R} \cap Q} \cup\brak{\brak{\sim Q} \cap \brak{\brak{\sim P} \cup \brak{\sim R}}}$
\item $\brak{\brak{P \cap R} \cap Q} \cap \brak{\brak{\sim Q} \cap \brak{\brak{\sim P} \cup \brak{\sim R}}}$
\item $\brak{\brak{P \cap \brak{\sim R}} \cap Q} \cup \brak{\brak{\sim Q} \cap \brak{\brak{\sim P} \cap R}}$
\end{enumerate}
\item Let $A = \cbrak{1,2,3,4,5,6,7}$.Define $B = \cbrak{T \subset A : \text{eitther } 1 \notin T \text{or } 2 \in T}$ and $C = \cbrak{T \subset A : T \text{the sum of all the elements of} T \text{ is a prime number}}$.Then the number of elements in the set $B \cup c$ is .
\item Let $f\brak{x}$ be a quadractic polynomial with leading coefficient $1$ such that $f\brak{0} = p,p \neq 0$. and $f\brak{1}=\frac{1}{3}$.If the equation $f\brak{x}=0$ and $fo fo fo fo f\brak{x} = 0$ have a common real root, then $f\brak{-3}$ is eqaul to .
\item Let $A = \myvec{1&a&a \\ 0&1&b \\ 0&0&1} ,a,b \in R$.If for some $n\in N,A^n = \myvec{1&48&2160 \\ 0&1&96 \\ 0&0&1}$ \\
then $n+a+b$ is equal to .
\item The sum of the maximum and minimum values of the function $f\brak{x} = |5x-7| + \sbrak{x^2+2x}$ in the interval $\sbrak{\frac{5}{4},2}$, where $\sbrak{t}$ is the greatest integer $\leq t$, is .
\item Let $y=y\brak{x}$ be the solution of the differential equation \\
$\frac{dy}{dx} = \frac{4y^3+2yx^2}{3xy^2+x^3},y\brak{1}=1$. \\
If for some $n \in N, y\brak{2} \in \lsbrak{n-1},\rbrak{n}$, then n is equal to .
\item let $f$ be twice differentiable function on R.If $f'\brak{0} = 4$ and ${f\brak{x}+ \int_{0}^{x} \brak{x-1}f'\brak{t}dt =  \brak{e^{2x} + e^{-2x}}\cos{2x} + \frac{2}{a} x}$,then $\brak{2a+5}^5a^2$ is equal to .
\item Let $a_n=\int_{-1}^{n}\brak{1+\frac{x}{2}+\frac{x^2}{3}+\dots+\frac{x^{n-1}}{n}}dx$//
for every $n \in N$.Then the sum of all the elments of the set $\cbrak{n \in N: a_n \in \brak{2,30}}$ is .
\item If the circles $x^2+y^2+6x+8y+16=0$ and ${x^2+y^2+2\brak{3-\sqrt{x}}x+2\brak{4-\sqrt{6}}y = k+6\sqrt{3} +8\sqrt{6},k\ge 0}$,touch internally at the point 
$P\brak{\alpha,\beta}$, then $\brak{\alpha+\sqrt{3}}^2+\brak{\beta+\sqrt{6}}^2$ is equal to .
\item Let the area enclosed by the x-axis, and the tangent and normal drawn to the curve $4x^3-3xy^2+6x^2-5xy-8y^2+9x+14 = 0$ at the point $\brak{-2,3}$ be A. Then $8A$ is equal to .
\item Let $x=\sin\brak{2\tan^{-1}\alpha}$ and $y=\sin\brak{\frac{1}{2}\tan^{-1}\frac{4}{3}}$.If ${S=\cbrak{\alpha \in R : y^2=1-x}}$, then ${\sum_{\alpha \in S} 16\alpha^3}$ is equal to .
\end{enumerate}
\end{document}


