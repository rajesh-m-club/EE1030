%iffalse
\let\negmedspace\undefined
\let\negthickspace\undefined
\documentclass[journal,12pt,twocolumn]{IEEEtran}
\usepackage{cite}
\usepackage{amsmath,amssymb,amsfonts,amsthm}
\usepackage{algorithmic}
\usepackage{multicol}
\usepackage{graphicx}
\usepackage{textcomp}
\usepackage{xcolor}
\usepackage{txfonts}
\usepackage{listings}
\usepackage{enumitem}
\usepackage{mathtools}
\usepackage{gensymb}
\usepackage{comment}
\usepackage[breaklinks=true]{hyperref}
\usepackage{tkz-euclide} 
\usepackage{listings}
\usepackage{gvv}                                        
%\def\inputGnumericTable{}                                 
\usepackage[latin1]{inputenc}                                
\usepackage{color}                                            
\usepackage{array}                                            
\usepackage{longtable}                                       
\usepackage{calc}                                             
\usepackage{multirow}                                         
\usepackage{hhline}                                           
\usepackage{ifthen}                                           
\usepackage{lscape}
\usepackage{tabularx}
\usepackage{array}
\usepackage{float}

\newtheorem{theorem}{Theorem}[section]
\newtheorem{problem}{Problem}
\newtheorem{proposition}{Proposition}[section]
\newtheorem{lemma}{Lemma}[section]
\newtheorem{corollary}[theorem]{Corollary}
\newtheorem{example}{Example}[section]
\newtheorem{definition}[problem]{Definition}
\newcommand{\BEQA}{\begin{eqnarray}}
\newcommand{\EEQA}{\end{eqnarray}}
\newcommand{\define}{\stackrel{\triangle}{=}}
\theoremstyle{remark}
\newtheorem{rem}{Remark}

% Marks the beginning of the document
\begin{document}
\bibliographystyle{IEEEtran}
\vspace{3cm}

\title{Assignment-1}
\author{ee24btech11043-Muura Rajesh Kumar Reddy}
\maketitle
\newpage
\bigskip

\renewcommand{\thefigure}{\theenumi}
\renewcomnand{\thetable}{\theenumi}

		\begin{enumerate}[start=6]
	\item Let $f$ be a one-one function with domain $\{x,y,z\}$ and range $\{1,2,3\}$. It is given that exactly one of the following statements is true and the remaining two are false ${f(x)=1}$,${f(y)\neq1}$,$f(z)\neq2$ determine $f^{-1}(1)$. \hfill${(1981 - 2 Marks)}$
	\item Let $R$ be the set of real numbers and $f:R \to R$ be such that for all $x$ and $y$ in $R$ ${|f(x)-f(y)| \le |x-y|^3}$ .Prove that $f(x)$ is a constant. \hfill${(1988 - 2 Marks)}$
	\item Find the natural number $'a'$ for which ${\sum\limits_{k=1}^nf(a+k)=16(2^n -1)}$, where the function $'f'$ satisfies the relation ${f(x+y)=f(x)f(y)}$ for all natural numbers $x,y$ and further ${f(1)=2}$. \hfill${(1992- 6 Marks)}$
	\item Let $\{x\}$ and $[x]$ denotes the fractional and integral part of a real number $x$ respectively. Solve $4\{x\}=x+[x]$. \hfill${(1994- 4 Marks)}$
	\item A function $f:IR\to IR$, where $IR$ is the set of real numbers, is defined by ${f(x)=\frac{\alpha x^2 +6x -8}{\alpha +6x-8x^2}}$. Find the interval of values $\alpha$ for which $f$ is onto. Is the function one-to-one for $\alpha=3$? Justify your answer. \hfill${(1996- 5 Marks)}$
	\item Let ${f(x)=Ax^2+Bx+C}$ where $A,B,C$ are real numbers. Prove that if $f(x)$ is an inteher whenever $x$ is an integer, then the numbers $2A,A+B$ and $C$ are all integers. Conversly, prove that if the numbers $2A,A+B$ and $C$ are all integers then $f(x)$ is an integer whenever $x$ is an integer. \hfill${(1998- 8 Marks)}$ 
		 \end{enumerate}
	\onecolumn
	\section*{F Match the following}
	\begin{enumerate}
		\item[1.] Let the function defined in column I have domain $(-\frac{\pi}{2},\frac{\pi}{2})$ and range $(-\infty,\infty)$ \hfill${(1992-2 Marks)}$
	\end{enumerate}
			\begin{multicols}{2} 
				\section*{Column I}
				\begin{enumerate}[label=(\Alph*)]
					\item $1+2x$
					\item $tanx$
				\end{enumerate}
				\columnbreak
				 \section*{Column II}
				\begin{enumerate}[label=(\alph*),start=16]
					\item onto but not one-one
					\item one-one but not onto
					\item one-one and onto
					\item neither one-one nor onto
				\end{enumerate}
			\end{multicols}
			\begin{enumerate}[start=2]
				\item Let ${f(x)=\frac{x^2-6x+5}{x^2-5x+6}}$ \hfill${(2007-6 Marks)}$ \\
				 Match of expressions/statements in Column I with expressions/statements in Column II and indicate your answer by darkening the appropriate bubbles in the $4*4$ matrix given in the ORS.
			\end{enumerate}
			              \begin{multicols}{2}
					      \section*{ Column I}
						\begin{enumerate}[label=(\Alph*)]
							\item If $-1<x<1$, then $f(X)$ satisfies
							\item If $1<x<2$, then $f(x)$ satisfies
							\item If $3<x<5$, then $f(x)$ satisfies
							\item If $x>5$, then $f(x)$ satisfies
						\end{enumerate}
						\columnbreak
						\section{ Column II}
						\begin{enumerate}[label=(\alph*) ,start=16]
							\item $0<f(x)<1$
							\item $f(x)<0$
							\item $f(x)>0$
							\item $f(x)<1$
						\end{enumerate}
					\end{multicols}
         \begin{tabular}{l|l}
\\
\hline
This section contains $4$ questions. Each questions has $2$ matching lists: LIST-I and LIST-II. Four options are given\\representing matching of elements from LIST-I and LIST-II. Only one of these four option corresponding to a\\correct matching.
\\
\hline
\end{tabular}
		 \begin{enumerate}[start=3]
	\item Let $E_1=\{x\in R:x\neq1$ and $\frac{x}{x-1}>0\}$ and $E_2=\{x\in E_1:\sin^{-1}(\log_e(\frac{x}{x-1}))$ is a real number $\}$.\\ 
		(Here, the inverse trigonometric function $\sin^{-1}x$ assumes values in $[-\frac{\pi}{2},\frac{\pi}{2})$.\\
		Let $f:E_1\to R$ be the function defined by $f(x)=\log_e(\frac{x}{x-1})$ and $g:E_2\to R$ be the function defined by $g(x)=\sin^{-1}(\log_e(\frac{x}{x-1}))$
		  \hfill${(JEE Adv.2018)}$
                 \end{enumerate}
\begin{multicols}{2}
	\section{LIST-I}
	\begin{enumerate}[label=\Alph*, start=16]
	\item The range of $f$ is
	\item The range of $g$ contains
	\item The domain of $f$ contains
	\item The domain of $g$ is 
\end{enumerate}
\columnbreak
	\section{LIST-II}
\begin{enumerate}
	\item $(-\infty,\frac{1}{1-e}]\cup[\frac{e}{e-1},\infty)$
	\item $(0,1)$
	\item $[-\frac{1}{2},\frac{1}{2}]$
	\item $(-\infty,0)\cup(0,\infty)$
	\item $(-\infty,\frac{e}{e-1}]$
	\item $(-\infty,0)\cup(\frac{1}{2},\frac{e}{e-1})$
\end{enumerate}
\end{multicols}
The correct option is:
\begin{multicols}{2}
	\begin{enumerate}
		\item[(a)] P$\to4$; Q$\to2$; R$\to1$; S$\to1$
		\item[(c)] P$\to4$; Q$\to2$; R$\to1$; S$\to6$
	\end{enumerate}
	\columnbreak
	\begin{enumerate}
		\item[(b)] P$\to3$; Q$\to3$; R$\to6$; S$\to5$
		\item[(d)] P$\to4$; Q$\to3$; R$\to6$; S$\to5$
	\end{enumerate}
\end{multicols}
\twocolumn
	    \section*{I Integer Value Correct type}
		   \begin{enumerate}
			   \item Let $f:[0,4\pi]\to[0,\pi]$ be defined by $f(x)=\cos^{-1}(\cos x)$. The number of points $x\in[0,4\pi]$ satisfying the equation ${f(x)=\frac{10-x}{10}}$ is \hfill${(JEE Adv.2014)}$
			   \item The  value  of  $((\log_29)^2)^{\frac{1}{\log_2(\log_29)}}\times (\sqrt7)^{\frac{1}{log_47}}$  is .
				   \hfill${(JEEAdv.2018)}$
			   \item Let X be a set with exactly $5$ elements and Y be a set with exactly $7$ elements. If $\alpha$ is the number of one-one functions from X to Y and $\beta$ is the number of onto functions from Y to X, then the value of $\frac{1}{5!}(\beta-\alpha)$ is ${(JEEAdv.2018)}$
		   \end{enumerate}
   \section*{Section-B JEE Main/ AIEEE}
	  \begin{enumerate}
		  \item The domain of $\sin^{-1}[\log_3(\frac{x}{3}]$ is 
			  \hfill$|2002|$
			  \begin{enumerate}
		  \item $[1,9]$    
		  \item $[-1,9]$    
		  \item $[-9,1]$     
		  \item $[-9,1]$
			  \end{enumerate}
		  \item The function $f(x)=log(x+\sqrt{x^2+1})$, is 
			  \hfill$|2003|$
			  \begin{enumerate}[label=(\alph*)]
		  \item neither an even nor an odd function
		  \item an even function
		  \item an odd function
		  \item a periodic function.
			  \end{enumerate}
		  \item Domain of definition of the function $f(x)=\frac{3}{4-x^2}+\log_{10}(x^3-x)$ ,is 
			  \hfill$|2003|$
			  \begin{enumerate}[label=(\alph*)]
		  \item $(-1,0)\cup(1,2)\cup(2,\infty)$     
		  \item $(a,2)$
		  \item $(-1,0)\cup(a,2)$                   
		  \item $(1,2)\cup(2,\infty)$.
			  \end{enumerate}
	  \end{enumerate}
\end{document}

