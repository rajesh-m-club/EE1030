\let\negmedspace\undefined
\let\negthickspace\undefined
\documentclass[journal]{IEEEtran}
\usepackage[a5paper, margin=10mm, onecolumn]{geometry}
%\usepackage{lmodern} % Ensure lmodern is loaded for pdflatex
\usepackage{tfrupee} % Include tfrupee package

\setlength{\headheight}{1cm} % Set the height of the header box
\setlength{\headsep}{0mm}     % Set the distance between the header box and the top of the text

\usepackage{gvv-book}
\usepackage{gvv}
\usepackage{cite}
\usepackage{amsmath,amssymb,amsfonts,amsthm}
\usepackage{algorithmic}
\usepackage{graphicx}
\usepackage{textcomp}
\usepackage{xcolor}
\usepackage{txfonts}
\usepackage{listings}
\usepackage{enumitem}
\usepackage{mathtools}
\usepackage{gensymb}
\usepackage{comment}
\usepackage[breaklinks=true]{hyperref}
\usepackage{tkz-euclide} 
\usepackage{listings}
% \usepackage{gvv}                                        
\def\inputGnumericTable{}                                 
\usepackage[latin1]{inputenc}                                
\usepackage{color}                                            
\usepackage{array}                                            
\usepackage{longtable}                                       
\usepackage{calc}                                             
\usepackage{multirow}                                         
\usepackage{hhline}                                           
\usepackage{ifthen}                                           
\usepackage{lscape}
\begin{document}

\bibliographystyle{IEEEtran}
\vspace{3cm}

\title{10.3.6.1.2}
\author{EE24BTECH11043 - Murra Rajesh Kumar Reddy}
% \maketitle
% \newpage
% \bigskip
{\let\newpage\relax\maketitle}

\renewcommand{\thefigure}{\theenumi}
\renewcommand{\thetable}{\theenumi}
\setlength{\intextsep}{10pt} % Space between text and floats

\numberwithin{figure}{enumi}
\renewcommand{\thetable}{\theenumi}

\section*{Question}
    Solve the following pair of linear equations using LU decomposition:\\
\textbf{Solution:}\\
\begin{align}
    \frac{2}{\sqrt{x}}+ \frac{3}{\sqrt{y}} = 2 \\
    \frac{4}{\sqrt{x}} - \frac{9}{\sqrt{y}} = -1
\end{align}

First, we rewrite the question as a system of linear equations.
\begin{align}
    x_1 &\implies \frac{1}{\sqrt{x}} \\
    x_2 &\implies \frac{1}{\sqrt{y}}
\end{align}

Converting into matrix form, we get:
\begin{align}
    \myvec{2 & 3\\ 4 & -9}\myvec{x_1 \\ x_2} &= \myvec{2 \\-1} \\
    \vec{A}x &= \vec{b}
\end{align}
To solve the above equation, we apply LU decomposition to matrix \(\vec{A}\).

\subsection*{Step 2: LU Factorization Using Update Equations}
Given a matrix $\vec{A}$ of size $n \times n$, LU decomposition is performed row by row and column by column. The update equations are as follows:

\textbf{Step-by-Step Procedure:}\\
1. \textbf{Initialization:}  
   - Start by initializing $\vec{L}$ as the identity matrix $\vec{L} = \vec{I}$ and $\vec{U}$ as a copy of $\vec{A}$.

2. \textbf{Iterative Update:}  
   - For each pivot $k = 1, 2, \ldots, n$:  
     - Compute the entries of $\vec{U}$ using the first update equation.  
     - Compute the entries of $\vec{L}$ using the second update equation.  

3. \textbf{Result:}  
   - After completing the iterations, the matrix $\vec{A}$ is decomposed into $\vec{L} \cdot \vec{U}$, where $\vec{L}$ is a lower triangular matrix with ones on the diagonal, and $\vec{U}$ is an upper triangular matrix.

\subsection*{1. Update for $U_{k,j}$ (Entries of $\vec{U}$)}
For each column $j \geq k$, the entries of $\vec{U}$ in the $k$-th row are updated as:
\[
U_{k,j} = A_{k,j} - \sum_{m=1}^{k-1} L_{k,m} \cdot U_{m,j}, \quad \text{for } j \geq k.
\]
This equation computes the elements of the upper triangular matrix $\vec{U}$ by eliminating the lower triangular portion of the matrix.

\subsection*{2. Update for $L_{i,k}$ (Entries of $\vec{L}$)}
For each row $i > k$, the entries of $\vec{L}$ in the $k$-th column are updated as:
\[
L_{i,k} = \frac{1}{U_{k,k}} \left( A_{i,k} - \sum_{m=1}^{k-1} L_{i,m} \cdot U_{m,k} \right), \quad \text{for } i > k.
\]

LU Factorizing \(\vec{A}\), we get:
\begin{align}
    \vec{A} &= \myvec{1 & 0\\2 & 1}\myvec{2 & 3\\0 & -15}, \\
    \vec{L} &= \myvec{1 & 0\\2 & 1}, \\
    \vec{U} &= \myvec{2 & 3\\0 & -15}
\end{align}
The solution can now be obtained as:
\begin{align}
    \myvec{1 & 0\\2 & 1}\myvec{y_1 \\ y_2} &= \myvec{2 \\-1}
\end{align}
Solving for \(y\), we get:
\begin{align}
    \myvec{y_1 \\ y_2} = \myvec{2 \\ -5}
\end{align}
Now, solving for \(x\) via back substitution:
\begin{align}
    \myvec{2 & 3\\0 & -15}\myvec{x_1 \\ x_2} &= \myvec{2 \\ -5}
\end{align}
\begin{align}
    x_2 = \frac{1}{3}, \\
    2 x_1 + 3 x_2 = 2 \implies x_1 = \frac{1}{2}
\end{align}
Thus, the solution is:
\begin{align}
    \frac{1}{\sqrt{x}} = \frac{1}{2}, \; \frac{1}{\sqrt{y}} = \frac{1}{3} \\
    x = 4, \; y = 9
\end{align}
\begin{figure}[H]
    \centering
    \includegraphics[width=\columnwidth]{figs/fig.png}
\end{figure}
\end{document}
