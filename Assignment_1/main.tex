\let\negmedspace\undefined
\let\negthickspace\undefined
\documentclass[journal,12pt,onecolumn,article]{IEEEtran}
\usepackage{cite}
\usepackage{color,soul}
\usepackage{amsmath,amssymb,amsfonts,amsthm}
\usepackage{algorithmic}
\usepackage{graphicx}
\usepackage{textcomp}
\usepackage{xcolor}
\usepackage{txfonts}
\usepackage{listings}
\usepackage{enumitem}
\usepackage{mathtools}
\usepackage{gensymb}
\usepackage{comment}
\usepackage[breaklinks=true]{hyperref}
\usepackage{tkz-euclide} 
\usepackage{listings}
\usepackage{multicol}
\usepackage{gvv}       
\usepackage[dvipsnames]{xcolor}
\def\inputGnumericTable{}                                
\usepackage[latin1]{inputenc}                            
\usepackage{color}                                       
\usepackage{array}                                       
\usepackage{longtable}                                   
\usepackage{calc}                                        
\usepackage{multirow}                                    
\usepackage{hhline}                                      
\usepackage{ifthen}                                      
\usepackage{lscape}
\newtheorem{theorem}{Theorem}[section]
\newtheorem{problem}{Problem}
\newtheorem{proposition}{Proposition}[section]
\newtheorem{lemma}{Lemma}[section]
\newtheorem{corollary}[theorem]{Corollary}
\newtheorem{example}{Example}[section]
\newtheorem{definition}[problem]{Definition}
\newcommand{\BEQA}{\begin{eqnarray}}
\newcommand{\EEQA}{\end{eqnarray}}
\newcommand{\define}{\stackrel{\triangle}{=}}
\theoremstyle{remark}
\newtheorem{rem}{Remark}
\begin{document}
\bibliographystyle{IEEEtran}
\vspace{3cm}
\title{ASSIGNMENT-1}
\author{EE24BTECH11043 - Murra Rajesh Kumar Reddy}
\maketitle
\bigskip
\begin{enumerate}
	\item Let $f$ be a one-one function with domain $\cbrak{x,y,z}$ and range $\cbrak{1,2,3}$. It is given that exactly one of the following statements is true and the remaining two are false ${f\brak{x}=1},{f\brak{y}\neq1},{f\brak{z}\neq2}$ determine $f^{-1}\brak{1}$. \hfill(1981 - 2 Marks)
	\item Let $R$ be the set of real numbers and $f:R \to R$ be such that for all $x$ and $y$ in $R$ ${|f\brak{x}-f\brak{y}| \le |x-y|^3}$ .Prove that $f\brak{x}$ is a constant. \hfill(1988 - 2 Marks)
	\item Find the natural number $'a'$ for which ${\sum\limits_{k=1}^nf\brak{a+k}=16\brak{2^n -1}}$, where the function $'f'$ satisfies the relation ${f\brak{x+y}=f\brak{x}f\brak{y}}$ for all natural numbers $x, y$ and further ${f\brak{1}=2}$. \hfill(1992- 6 Marks)
	\item Let $\cbrak{x}$ and $\sbrak{x}$ denotes the fractional and integral part of a real number $x$ respectively. Solve ${4\cbrak{x}=x+\sbrak{x}}$. \hfill(1994- 4 Marks)
	\item A function $f:IR\to IR$, where $IR$ is the set of real numbers, is defined by ${f\brak{x}=\frac{\alpha x^2 +6x -8}{\alpha +6x-8x^2}}$. Find the interval of values $\alpha$ for which $f$ is onto. Is the function one-to-one for $\alpha=3$? Justify your answer. \hfill(1996- 5 Marks)
	\item Let ${f\brak{x}=Ax^2+Bx+C}$ where $A,B,C$ are real numbers. Prove that if $f\brak{x}$ is an integer whenever $x$ is an integer, then the numbers $2A, A+B$ and $C$ are all integers. Conversly, prove that if the numbers $2A, A+B$ and $C$ are all integers then $f\brak{x}$ is an integer whenever $x$ is an integer. \hfill(1998- 8 Marks)
\end{enumerate}
	\section*{F Match the Following}
	\begin{enumerate}
		\item Let the function defined in column I have domain $\brak{-\frac{\pi}{2},\frac{\pi}{2}}$ and range $\brak{-\infty,\infty}$ \hfill(1992-2 Marks)
			\begin{multicols}{2} 
				\section*{Column I}
				\begin{enumerate}[label=(\Alph*)]
					\item $1+2x$
					\item $\tan x$
				\end{enumerate}
				\columnbreak
				 \section*{Column II}
				\begin{enumerate}[label=(\alph*),start=16]
					\item onto but not one-one
					\item one-one but not onto
					\item one-one and onto
					\item neither one-one nor onto
				\end{enumerate}
			\end{multicols}
		\item Let $f\brak{x}=\frac{x^2-6x+5}{x^2-5x+6}$ \hfill(2007-6 marks)
					\\ Match of expressions/statements in Column I with expressions/statements in Column II and indicate your answer by darkening the appropriate bubbles in the $4\times4$ matrix given in the ORS.
	
			              \begin{multicols}{2}
					      \section*{ Column I}
						\begin{enumerate}[label=(\Alph*)]
							\item If $-1<x<1$, then $f\brak{x}$ satisfies
							\item If $1<x<2$, then $f\brak{x}$ satisfies
							\item If $3<x<5$, then $f\brak{x}$ satisfies
 							\item If $x>5$, then $f\brak{x}$ satisfies
						\end{enumerate}
						\columnbreak
						\section*{ Column II}
						\begin{enumerate}[label=(\alph*) ,start=16]
							\item $0<f\brak{x}<1$
							\item $f\brak{x}<0$
							\item $f\brak{x}>0$
							\item $f\brak{x}<1$
						\end{enumerate}
					\end{multicols}
         \begin{tabular}{l|l}
\\
\hline
This section contains $4$ questions. Each questions has $2$ matching lists: LIST-I and LIST-II. Four options are given\\representing matching of elements from LIST-I and LIST-II. Only one of these four option corresponding to a\\correct matching.
\\
\hline
        \end{tabular}
			 \item Let $E_1=\cbrak{x\in R:x\neq1}$ and $\frac{x}{x-1}>0$ and 
				 $E_2=\cbrak{x\in E_1:\sin^{-1}\brak{\log_e\brak{\frac{x}{x-1}}}\text{ is a real number}}$.\\
					$ \brak{\text{Here, the inverse trigonometric function} \sin^{-1}x \text{assumes values in} \sbrak{-\frac{\pi}{2},\frac{\pi}{2}}}$.\\
			
			Let $f:E_1\to R$ be the function defined by $f\brak{x}=\log_e\brak{\frac{x}{x-1}}$ and $g:E_2\to R$ be the function defined by $g\brak{x}=\sin^{-1}\brak{\log_e\brak{\frac{x}{x-1}}}$
			\hfill(JEE Adv.2018)
\begin{multicols}{2}
	\section*{LIST-I}
	\begin{enumerate}[label=(\Alph*), start=16]
	\item The range of $f$ is
	\item The range of $g$ contains
	\item The domain of $f$ contains
	\item The domain of $g$ is 
\end{enumerate}
\columnbreak
	\section*{LIST-II}
\begin{enumerate}
	\item $\lbrak{-\infty},\rsbrak{\frac{1}{1-e}}\cup\lsbrak{\frac{e}{e-1}},\rbrak{\infty}$
	\item $\brak{0,1}$
	\item $\sbrak{-\frac{1}{2},\frac{1}{2}}$
	\item $\brak{-\infty,0}\cup\brak{0,\infty}$
	\item $\lbrak{-\infty},\rsbrak{\frac{e}{e-1}}$
	\item $\brak{-\infty,0}\cup\brak{\frac{1}{2},\frac{e}{e-1}}$
\end{enumerate}
\end{multicols}
The correct option is:
\begin{multicols}{2}
		\begin{enumerate}
		\item P$\to4$; Q$\to2$; R$\to1$; S$\to1$
		\item P$\to4$; Q$\to2$; R$\to1$; S$\to6$
	\columnbreak
		\item P$\to3$; Q$\to3$; R$\to6$; S$\to5$
		\item P$\to4$; Q$\to3$; R$\to6$; S$\to5$
	\end{enumerate}
\end{multicols}
\end{enumerate}
	    \section*{I Integer Value Correct type}
		   \begin{enumerate}
			   \item Let $f:\sbrak{0,4\pi}\to\sbrak{0,\pi}$ be defined by $f\brak{x}=\cos^{-1}\brak{\cos x}$. The number of points $x\in\sbrak{0,4\pi}$ satisfying the equation  ${f\brak{x}=\frac{10-x}{10}}$ is \hfill(JEE Adv.2014)
			   \item The  value  of $\brak{\brak{\log_29}^2}^{\frac{1}{\log_2\brak{\log_29}}}\times \brak{\sqrt7}^{\frac{1}{\log_47}}$ is .
				   \hfill(JEEAdv.2018)
			   \item Let $X$ be a set with exactly $5$ elements and $Y$ be a set with exactly $7$ elements. If $\alpha$ is the number of one-one functions from $X$ to $Y$ and $\beta$ is the number of onto functions from $Y$ to $X$, then the value of $\frac{1}{5!}\brak{\beta-\alpha}$ is \hfill(JEEAdv.2018)
		   \end{enumerate}
   \section*{Section-B JEE Main/ AIEEE}
	  \begin{enumerate}
		  \item The domain of $\sin^{-1}\sbrak{\log_3\brak{\frac{x}{3}}}$ is 
			  \hfill$|2002|$
			  \begin{enumerate}
				  \item $\sbrak{1,9}$    
				  \item $\sbrak{-1,9}$    
				  \item $\sbrak{-9,1}$     
				  \item $\sbrak{-9,1}$
			  \end{enumerate}
		  \item The function $f\brak{x}=\log\brak{x+\sqrt{x^2+1}}$, is 
			  \hfill$|2003|$
			  \begin{enumerate}
		  \item neither an even nor an odd function
		  \item an even function
		  \item an odd function
		  \item a periodic function.
			  \end{enumerate}
		  \item Domain of definition of the function ${f\brak{x}=\frac{3}{4-x^2}+\log_{10}\brak{x^3-x}}$ is 
			  \hfill$|2003|$
			  \begin{enumerate}
				  \item $\brak{-1,0}\cup\brak{1,2}\cup\brak{2,\infty}$     
				  \item $\brak{a,2}$
				  \item $\brak{-1,0}\cup\brak{a,2}$                   
				  \item $\brak{1,2}\cup\brak{2,\infty}$.
			  \end{enumerate}
	  \end{enumerate}
\end{document}

